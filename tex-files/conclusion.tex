\section{Conclusion}
\label{sec:conclusion}

Throughout the chapters of this thesis, we have seen that the \WSP\ can be studied in many different variations. We showed that all of those variations are \NP-complete. Nevertheless, we managed to solve the problem efficiently on certain classes of graphs, including paths, trees, series-parallel, and decomposable graphs. This was achieved by constructing a polynomial time dynamic programming algorithm. Also, several integer programming formulations for the problem and its variations were given.\medskip

Furthermore, we showed that the \WSP\ and its variations are \NP-complete to approximate within a constant factor. Unfortunately, we were not able to develop an approximation algorithm, but we proposed two heuristic procedures.  Moreover, a preprocessing scheme for reducing instances and a postprocessing procedure for improving heuristic solutions were presented.\medskip

In the empirical part of this thesis, the dynamic programming algorithm and the spanning tree heuristic turned out to be a lot faster than Gurobi. Besides the practical use of the procedures, it is a nice theoretical result that in some cases we can solve the \WSP\ and the \WISP\ in polynomial or even linear time.\medskip

It remains open to give an approximation algorithm for the \WSP\ with a provable ratio. In the time given to write this thesis, we did not manage to achieve this.