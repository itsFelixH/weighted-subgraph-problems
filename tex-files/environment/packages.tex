% Packages

% Provides the algorithm environment.
\usepackage{algorithm}
% Basic math packages
\usepackage{amsmath,amssymb,amsthm}
\usepackage{latexsym}
% For declaring MathOperators
\usepackage{amsopn}
% A reimplementation of the bibliographic facilities. Here used to use the 'alphabetic' citation style
\usepackage[backend = biber, style = alphabetic]{biblatex}
\addbibresource{bibliography.bib}
% Provides options to change the style of captions. Also avoids hyperref to jump to the caption of a float, but to the top of the float instead (like package hypcap).
\usepackage{caption}
% For adding subcaptions to plots/figures
\usepackage{subcaption}
% For defining new colors
\usepackage{color}
% Provides \mathds for double struck letters and digits
\usepackage{dsfont}
% Provides options to manually change the layout of enumerate or itemize environments
\usepackage{enumitem}
% For layout settings concerning the geometry of the page
\usepackage{geometry}
% For hyperlinks in references. Includes hyperlinks automatically for all references
\usepackage{hyperref}
% For declaring new theorem styles. Produces warnings if loaded before hyperref when cleverref is used.
\usepackage{amsthm}
% Adds the number of the current section to the number of an equation (e.g. (3.x) in section 3. Produced warnings when used after loading cleveref
\numberwithin{equation}{section}
\numberwithin{figure}{section}
\numberwithin{table}{section}
\numberwithin{algorithm}{section}
% Costumizing sections
\usepackage{titlesec}
% Provides the algorithmic environment for using pseudocode. 'noend' suppresses the 'end' statement for loops etc. Has to be loaded after hyperref and before cleveref to work without warnings.
\usepackage[noend]{algpseudocode}
% For adding an appendix
\usepackage[titletoc]{appendix}
% For using better references, for instance references to enumerate items in an environment. Has to be loaded after hyperref
\usepackage{cleveref}
% Sets the encoding of the input
\usepackage[utf8]{inputenc}
% T1 font encoding, better than default encoding
\usepackage[T1]{fontenc}
\usepackage{lmodern}
% Rechtschreibung
\usepackage[english]{babel}
% Ensures that quotes are typeset according to the rules of the main language. Should be loaded after inputenc
\usepackage{csquotes}
% For alignment in matrices
\usepackage{mathtools}
% Improves appearence of the text by adjusting whitespace between words and other things. Also provides some options.
\usepackage[final]{microtype}
% For using multiple columns of text
\usepackage{multicol}
% For fractions in the style of 1/4.
\usepackage{nicefrac}
% Avoids widows and orphans (single lines at the top or bottom of a page) by shifting it to the preceding or subsequent page. Can cause more unequal length of pages.
\usepackage[all]{nowidow}
% Avoids warnings with older packages like listings, floatrow, float, algorithm
\usepackage{scrhack}
% For changing the line distance
\usepackage{setspace}
% For text markup like highlighting, crossing, bold text, etc.
\usepackage{soul}
% For using tikzpictures for graphics.
\usepackage{tikz}
\usetikzlibrary{arrows}
\usetikzlibrary{automata}
\usetikzlibrary{trees}
\usetikzlibrary{graphs}
\usetikzlibrary{positioning}
\usetikzlibrary{shapes}
\usetikzlibrary{shapes.arrows}
\usetikzlibrary{calc}
\usetikzlibrary{decorations.pathreplacing}
\usetikzlibrary{decorations.pathmorphing}
% For plotting diagrams
\usepackage{pgfplots}
% for drawing trees
\usepackage{forest}
% For using figures wrapped by text
\usepackage{wrapfig}
% Allows block commenting
\usepackage{verbatim}
% Optional arguments to the \includegraphics command
\usepackage{graphicx}
\graphicspath{{graphics/}}
% for appendix
\usepackage[titletoc]{appendix}
% for programming code in Latex
\usepackage{listings}
\lstset{basicstyle=\footnotesize, tabsize=4, language=Python}
% Improved interface for floating objects
\usepackage{float}
% Customise captions in floating environments
\usepackage{caption}
% Extends the array and tabular environments
\usepackage{array}
% an extension of tabular
\usepackage{tabularx}
% interface for environments
\usepackage{environ}
% Allows to use if in \newcommand
\usepackage{ifthen}
% For defining optimization problems
\usepackage[short]{optidef}
% Necessary to make hyperref, cleveref and algpseudocode work together properly
\makeatletter
\newcounter{algorithmicH}% New algorithmic-like hyperref counter
\let\oldalgorithmic\algorithmic
\renewcommand{\algorithmic}{%
\stepcounter{algorithmicH}% Step counter
\oldalgorithmic}% Do what was always done with algorithmic environment
\renewcommand{\theHALG@line}{ALG@line.\thealgorithmicH.\arabic{ALG@line}}
\makeatother