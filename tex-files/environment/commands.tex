% New commands

% Produces a well scaled := sign
\newcommand{\defeq}{\mathrel{\vcenter{\baselineskip0.45ex \lineskiplimit0pt
                     \hbox{\footnotesize.}\hbox{\footnotesize.}}}=}
% Produces a well scaled =: sign
\newcommand{\eqdef}{=\mathrel{\vcenter{\baselineskip0.45ex \lineskiplimit0pt
                     \hbox{\footnotesize.}\hbox{\footnotesize.}}}}

% Produces an empty page after a newpage (the newpage command has to be used before)
\newcommand{\emptypage}{\thispagestyle{empty}~\newpage}

% Manually produces the circle for the end of environments. The first command is for text mode the second for displaymode
\newcommand{\tendcirc}{\hfill\ensuremath{\mathscale{0.75}{\bigcirc}}}
\newcommand{\dendcirc}{\tag*{\ensuremath{\mathscale{0.75}{\bigcirc}}}}
										
% Produces smaller fractionals in display mode
\newcommand{\fracs}[2]{\raisebox{0pt}{$\frac{#1}{#2}$}}

% Produces a box wich is scaled (by a given factor) in math-mode.
\newcommand{\mathscale}[2]{\vcenter{\hbox{\text{\scalebox{#1}{$#2$}}}}}

% Adds only one line in an align* environment to the counter
\newcommand{\numberthis}{\addtocounter{equation}{1}\tag{\theequation}}

% Label with custom text
\newcommand{\textlabel}[2]{% 
	\edef\@currentlabel{#1}% Set target label 
	\phantomsection% Correct hyper reference link 
	#1\protect\label{#2}% Print and store label 
}

% For the opening of an optimization problem (appropriate whitespace included)
\newcommand{\progmin}{\min\hspace{0.5em}}
\newcommand{\progmax}{\max\hspace{0.5em}}

% For defining graphs
\newcommand{\graph}{\ensuremath{G=(V,R,\alpha,\omega)}}
\newcommand{\ugraph}{\ensuremath{G=(V,E,\gamma)}}

% For easier usage of number sets
\newcommand{\RR}{\mathbb{R}}
\newcommand{\NN}{\mathbb{N}}
\newcommand{\ZZ}{\mathbb{Z}}
\newcommand{\QQ}{\mathbb{Q}}
\newcommand{\BB}{\mathbb{B}}
\newcommand{\ONE}{\mathds{1}}

% For easier usage of complexity notations
\newcommand{\OO}{\ensuremath{\mathcal{O}}}
\newcommand{\PP}{\ensuremath{\mathcal{P}}}
\newcommand{\NP}{\ensuremath{\mathcal{NP}}}

% Easier definitions of vectors
\newcommand{\Vector}[2][x]{(#1_1,#1_2,\ldots,#1_{#2})} %x=\Vector[x]{5}

% Adds commands for argmin and argmax
\DeclareMathOperator*{\argmax}{argmax}
\DeclareMathOperator*{\argmin}{argmin}

% epsilon
\newcommand{\eps}{\varepsilon}

% Produces a nonitalic OP, LP, IP in brackets
\newcommand{\OP}{\text{OP}}
\newcommand{\LP}{\text{LP}}
\newcommand{\IP}{\text{IP}}

% Produces a nonitalic transposed T, or a -T respectively
\newcommand{\tran}{^\text{T}}
\newcommand{\mtran}{^{-\text{T}}}

% Nonitalic E for the expactation
\newcommand{\Exp}{\text{E}}

% Makes a non-italic d
\newcommand{\dif}{\text{d}}

% Makes a non-italic D for the Differential \difff produces a superscript 2 addtitionally
\DeclareMathOperator{\diff}{D\hspace{-0.12em}}
\DeclareMathOperator{\difff}{D^2\hspace{-0.12em}}

\DeclareMathOperator{\rank}{rank}

% Recession cone
\DeclareMathOperator{\rec}{rec}

% For colored highlighting
\newcommand{\hlc}[2][yellow]{{\sethlcolor{#1}\hl{#2}}}

\definecolor{darkgreen}{rgb}{0.1,0.5,0}
\definecolor{lightred}{rgb}{1,0.4,0.4}
\definecolor{lightyellow}{rgb}{0.97,0.97,0.55}

% Draws an irregular circle in tikz
\newcommand\irregularcircle[2]{% radius, irregularity
	\pgfextra {\pgfmathsetmacro\len{(#1)+rand*(#2)}}
	+(0:\len pt)
	\foreach \a in {10,20,...,350}{
		\pgfextra {\pgfmathsetmacro\len{(#1)+rand*(#2)}}
		-- +(\a:\len pt)
	} -- cycle
}


% document specific macros

% Macros
\newcommand{\WSP}{\textsc{Weighted Subgraph Problem}}
\newcommand{\minWSP}{\ensuremath{\text{\textsc{min-wsp}}}}
\newcommand{\maxWSP}{\ensuremath{\text{\textsc{max-wsp}}}}

\newcommand{\WISP}{\textsc{Weighted Induced Subgraph Problem}}
\newcommand{\minWISP}{\ensuremath{\text{\textsc{min-wisp}}}}
\newcommand{\maxWISP}{\ensuremath{\text{\textsc{max-wisp}}}}

\newcommand{\RWSP}{\textsc{Rooted Weighted Subgraph Problem}}
\newcommand{\minRWSP}{\ensuremath{\text{\textsc{min-rwsp}}}}
\newcommand{\maxRWSP}{\ensuremath{\text{\textsc{max-rwsp}}}}

\newcommand{\RWISP}{\textsc{Rooted Weighted Induced Subgraph Problem}}
\newcommand{\minRWISP}{\ensuremath{\text{\textsc{min-rwisp}}}}
\newcommand{\maxRWISP}{\ensuremath{\text{\textsc{max-rwisp}}}}

\newcommand{\WkSP}{\textsc{Weighted k-Subgraph Problem}}
\newcommand{\minWkSP}{\ensuremath{\text{\textsc{min-wksp}}}}
\newcommand{\maxWkSP}{\ensuremath{\text{\textsc{max-wksp}}}}

\newcommand{\WIkSP}{\textsc{Weighted Induced k-Subgraph Problem}}
\newcommand{\minWIkSP}{\ensuremath{\text{\textsc{min-wiksp}}}}
\newcommand{\maxWIkSP}{\ensuremath{\text{\textsc{max-wiksp}}}}

\newcommand{\BCWSP}{\textsc{Budget-Constraint Weighted Subgraph Problem}}
\newcommand{\maxBCWSP}{\ensuremath{\text{\textsc{bcwsp}}}}

\newcommand{\HPath}[2]{\ensuremath{H_{#1}^{\text{#2}}}}
\newcommand{\HSSP}[2]{\ensuremath{H_{#1}^{#2}}}
\newcommand{\WSPt}[2]{\ensuremath{t_{#1}^{#2}}}
\newcommand{\WSPT}[2]{\ensuremath{T_{#1}^{#2}}}
\newcommand{\WSPP}[2]{\ensuremath{P_{#1}^{#2}}}

\newcommand{\DkS}{\textsc{Densest k-Subgraph}}
\newcommand{\SAT}{\textsc{Satisfiability Problem}}

\newcommand{\cutr}{\ensuremath{(\mathrm{CUT}_r)}}
\newcommand{\cut}{\ensuremath{(\mathrm{CUT})}}
\newcommand{\flow}{\ensuremath{(\mathrm{FLOW})}}